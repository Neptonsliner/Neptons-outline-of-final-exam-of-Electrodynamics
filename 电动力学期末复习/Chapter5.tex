\documentclass[main.tex]{subfiles}
\usepackage[table,xcdraw]{xcolor}
\usepackage{bm}
\usepackage{wallpaper}
\usepackage{booktabs}
\usepackage{array}
\usepackage{ctex}
\usepackage{longtable}
\usepackage{physics}
\usepackage{fancyhdr}
\usepackage{graphicx} % Required for inserting images
\usepackage{float} %指定图片位置
\usepackage{caption}
\usepackage{pdfpages}
\usepackage{color}
\usepackage{rotating}
\usepackage{tabularray}
\usepackage{xeCJK}
\usepackage{verbatim}
\usepackage{type1cm}
\usepackage{tocloft}
\usepackage{multicol}
\usepackage{amssymb}
\usepackage{amsmath}
\usepackage{float}
\usepackage{wrapfig}
\usepackage{diagbox}
\usepackage{multirow}
\usepackage{subfigure}
\usepackage{svg}
\usepackage[hidelinks]{hyperref}
\usepackage[a4paper,left=2.85cm,right=2.85cm,top=2.54cm,bottom=2.54cm]{geometry}
\renewcommand{\normalsize}{\fontsize{14}{16}\selectfont}


\begin{document}

\chapter{电磁波的传播}
\section{平面电磁波}
\subsection{电磁场波动方程}
在没有自由电荷的真空中,描述电磁场运动规律的Maxwell方程为
\begin{align}
\label{Maxwell2}\left\{\begin{array}{l}
 \nabla \cdot \boldsymbol{D} = 0\\
 \nabla \times \boldsymbol{E} = -\frac{\partial \boldsymbol{B}}{\partial t}\\
\nabla \cdot  \boldsymbol{B} = 0\\
\nabla \times \boldsymbol{H} = \frac{\partial \boldsymbol{D}}{\partial t}
\end{array}\right.
\end{align}

取第二式的旋度,得到
\begin{align}
    &\nabla \times (\nabla \times \boldsymbol{E}) = -\frac{\partial }{\partial t} \nabla \times \boldsymbol{B} = -\mu_0 \varepsilon_0 \frac{\partial ^2 \boldsymbol{E}}{\partial t^2} = -\frac{1}{c^2}\frac{\partial ^2 \boldsymbol{E}}{\partial t^2}\\
    \label{xuandu5.1}&\nabla \times (\nabla \times \boldsymbol{E}) = \nabla(\nabla \cdot \boldsymbol{E}) - \nabla ^2 \boldsymbol{E} = -\nabla ^2 \boldsymbol{E}\ (\nabla \cdot \boldsymbol{E} = 0)
\end{align}

得到电磁波传播方程1
\begin{align}
    \nabla ^2 \boldsymbol{E} - \frac{1}{c^2}\frac{\partial ^2 \boldsymbol{E}}{\partial t^2} = 0
\end{align}

用同样的方法可以得到电磁波传播方程2
\begin{align}
    \nabla ^2 \boldsymbol{B} - \frac{1}{c^2}\frac{\partial ^2 \boldsymbol{B}}{\partial t^2} = 0
\end{align}

\subsection{时谐电磁波与其平面波解}
线性均匀介质中一定频率的电磁波,设角频率为$\omega$,电磁场可以表示为
\begin{align}
    \label{hanshiE}&\boldsymbol{E} (\boldsymbol{x}, t) = \boldsymbol{E}(\boldsymbol{x})\mathrm{e}^{-\mathrm{i}\omega t}\\
    &\boldsymbol{B} (\boldsymbol{x}, t) = \boldsymbol{B}(\boldsymbol{x})\mathrm{e}^{-\mathrm{i}\omega t}
\end{align}

将其代入Maxwell方程组\ref{Maxwell2}中,并应用线性均匀介质条件$\boldsymbol{D} = \varepsilon \boldsymbol{E}, \boldsymbol{B} = \mu \boldsymbol{H}$得到
\begin{align}
    \label{shixie1}&\nabla \times \boldsymbol{E} = \mathrm{i} \omega \mu \boldsymbol{H}\\
    \label{shixie2}&\nabla \times \boldsymbol{H} = -\mathrm{i}\omega \varepsilon \boldsymbol{E}\\
    &\nabla \cdot \boldsymbol{E} = 0\\
    &\nabla \cdot \boldsymbol{H} = 0
\end{align}

由于
\begin{align}
    &\nabla \cdot (\nabla \times \boldsymbol{E} )= \mathrm{i} \omega \mu \nabla \cdot \boldsymbol{H} = 0\\
    &\nabla \cdot (\nabla \times \boldsymbol{H}) = -\mathrm{i}\omega \varepsilon \nabla \cdot \boldsymbol{E} = 0\\
\end{align}

故只有式\ref{shixie1}与式\ref{shixie2}是独立的,后两项可由前两项导出。

同样地,我们对式\ref{shixie1}取旋度得到
\begin{align}
    \nabla \times (\nabla \times \boldsymbol{E}) = \mathrm{i}\omega \mu \nabla \times \boldsymbol{H} = \omega ^2 \mu \varepsilon \boldsymbol{E}
\end{align}

又由式\ref{xuandu5.1}得到
\begin{align}
    \nabla ^2 \boldsymbol{E} + \frac{\omega ^2}{v^2}\boldsymbol{E} = 0
\end{align}

令$\displaystyle \frac{\omega }{v} = k$,得到\textbf{Helmholtz方程}
\begin{align}
    \nabla ^2 \boldsymbol{E} +k^2 \boldsymbol{E} = 0
\end{align}

解出$\boldsymbol{E}$后,$\boldsymbol{B}$也可以由\ref{shixie1}解出:
\begin{align}
    \boldsymbol{B} = -\frac{\mathrm{i}}{\omega}\nabla \times \boldsymbol{E} = -\frac{\mathrm{i}}{k}\sqrt{\mu \varepsilon}\nabla \times \boldsymbol{E}
\end{align}

\textbf{平面波}:电磁波沿$x$方向传播,其场强在于$x$正交的方向上具有相同值.此时Helmholtz方程化为一维常微分方程
\begin{align}
    \frac{\mathrm{d}^2}{\mathrm{d} x^2}\boldsymbol{E}(\boldsymbol{x})+k^2 \boldsymbol{E}(\boldsymbol{x}) = 0
\end{align}

它的一个解是
\begin{align}
    \boldsymbol{E}(\boldsymbol{x}) = \boldsymbol{E}_0\mathrm{e}^{\mathrm{i}kx}
\end{align}

由式\ref{hanshiE}得到
\begin{align}
    \boldsymbol{E}(\boldsymbol{x},t) = \boldsymbol{E}_0\mathrm{e}^{\mathrm{i}kx-\omega t}
\end{align}

由$\nabla \cdot \boldsymbol{E} = 0$得到
\begin{align}
    \mathrm{i}k\boldsymbol{e}_x \cdot \boldsymbol{E} = 0 \Longrightarrow \boldsymbol{E}_x = 0
\end{align}

\subsection{电磁波的能量和能流}

在电磁波传播中,有关系
\begin{align}
    \boldsymbol{B} = \sqrt{\mu \varepsilon}\frac{\boldsymbol{k}}{k}\times \boldsymbol{E} = \sqrt{\mu \varepsilon}\boldsymbol{e}_k \times \boldsymbol{E}
\end{align}

即$\boldsymbol{E}$、$\boldsymbol{B}$、$\boldsymbol{k}$是三个相互正交的矢量,且$\boldsymbol{E}$和$\boldsymbol{B}$同相,振幅比为
\begin{align}
    \left|\frac{\boldsymbol{E}}{\boldsymbol{B}}\right| = \frac{1}{\sqrt{\mu \varepsilon}} = v
\end{align}

可以得到$\displaystyle \varepsilon E^2 = \frac{1}{\mu}B^2$.又由式\ref{nengliang}得到,在均匀线性介质中电磁场能量密度为
\begin{align}
    \mathscr{w} = \frac{1}{2} (\boldsymbol{E} \cdot \boldsymbol{D} + \boldsymbol{H} \cdot \boldsymbol{B}) = \frac{1}{2}\left(\varepsilon E^2+\frac{1}{\mu}B^2 \right)
\end{align}

故
\begin{align}
    \mathscr{w} = \varepsilon E^2 =\frac{1}{\mu}B^2
\end{align}

而由式\ref{nengliu}可以得到平面电磁波的能流密度
\begin{align}
    \boldsymbol{S} = \boldsymbol{E}\times \boldsymbol{H} = \sqrt{\frac{\varepsilon}{\mu}}\boldsymbol{E}\times (\boldsymbol{e}_k \times \boldsymbol{E}) = \sqrt{\frac{\varepsilon}{\mu}}E^2 \boldsymbol{e}_k = v \mathscr{w} \boldsymbol{e}_k
\end{align}

能量密度和能流密度的时间平均值
\begin{align}
    &\mathscr{w} = \frac{1}{2} \varepsilon E_0^2 =\frac{1}{2\mu}B_0^2\\
    &\boldsymbol{S} =\frac{1}{2} \sqrt{\frac{\varepsilon}{\mu}}E_0^2 \boldsymbol{e}_k
\end{align}

\section{反射和折射定律}

讨论时谐电磁波时,只需要考虑以下两个边界条件

\begin{align}
\label{tiaojian1}\left\{\begin{array}{l}
\boldsymbol{n} \cdot (\boldsymbol{E}_2 - \boldsymbol{E}_1) = 0 \\
\boldsymbol{n} \cdot (\boldsymbol{H}_2 - \boldsymbol{H}_1) = \alpha 
\end{array}\right.
\end{align}

\begin{wrapfigure}{r}{6cm}
	\centering	\includesvg[width=0.9\linewidth]{zheshefanshe.svg}
\end{wrapfigure}

设入射波、反射波和折射波的电场强度分别为$\boldsymbol{E}$、$\boldsymbol{E}'$和$\boldsymbol{E}''$,波矢量分别为$\boldsymbol{k}$、$\boldsymbol{k}'$和$\boldsymbol{k}''$,它们的平面波分别表示为:
\begin{align}
    &\boldsymbol{E} = \boldsymbol{E}_0\mathrm{e}^{\mathrm{i}(\boldsymbol{k} \cdot \boldsymbol{x} - \omega t)}\\
    &\boldsymbol{E}' = \boldsymbol{E}'_0\mathrm{e}^{\mathrm{i}(\boldsymbol{k}' \cdot \boldsymbol{x} - \omega t)}\\
    &\boldsymbol{E}'' = \boldsymbol{E}''_0\mathrm{e}^{\mathrm{i}(\boldsymbol{k}'' \cdot \boldsymbol{x} - \omega t)}
\end{align}

代入\ref{tiaojian1}第一式中得到

\begin{equation}
    \boldsymbol{n} \times (\boldsymbol{E}_0 \mathrm{e}^{\mathrm{i} \boldsymbol{k} \cdot \boldsymbol{x}} +\boldsymbol{E}'_0 \mathrm{e}^{\mathrm{i} \boldsymbol{k}' \cdot \boldsymbol{x}}) = \boldsymbol{n} \times \boldsymbol{E}''_0 \mathrm{e}^{\mathrm{i} \boldsymbol{k}'' \cdot \boldsymbol{x}}
\end{equation}

对界面$z = 0$成立.于是应当有
\begin{align}
    k_x = k'_x = k''_x,\ k_y = k'_y = k''_y = 0
\end{align}

$\theta$、$\theta '$和$\theta ''$代表入射角、反射角和折射角,有
\begin{align}
    &k_x = k \mathrm{sin}\theta\\
    &k'_x = k' \mathrm{sin}\theta '\\
    &k''_x = k'' \mathrm{sin}\theta ''\\
\end{align}

设电磁波在两介质中的相速度为$v_1$、$v_2$,则
\begin{align}
    \frac{\omega}{v_1} = k = k',\ \frac{\omega}{v_2} = k''
\end{align}

将波矢及其分量代入得到
\begin{align}
    \label{fanshezheshe}\theta = \theta ',\ \frac{\mathrm{sin}\theta}{\mathrm{sin}\theta''} = \frac{v_1}{v_2} = n_{21},\ v = \frac{c}{n}
\end{align}

这就是\textbf{反射和折射定律}

\section{有导体存在时电磁波的传播}
\subsection{导体内的电磁波}

在导体内部,$\rho _f = 0$,$\boldsymbol{J} = \sigma \boldsymbol{E}$.Maxwell方程组(式\ref{Maxwell3})中:
\begin{align}
    \nabla \times \boldsymbol{H} = \displaystyle \frac{\partial \boldsymbol{D}}{\partial t} + \sigma \boldsymbol{E}
\end{align}

对一定频率的电磁波,此时有
\begin{align}
    \left\{\begin{array}{l}
        \nabla \times \boldsymbol{E} = \mathrm{i} \omega \mu \boldsymbol{H}\\
        \nabla \times \boldsymbol{H} = -\mathrm{i}\omega \varepsilon \boldsymbol{E} + \sigma \boldsymbol{E} = -\mathrm{i}\omega \varepsilon ' \boldsymbol{E}\\
        \varepsilon ' = \varepsilon +\mathrm{i} \displaystyle \frac{\sigma}{\omega}
    \end{array}\right.
\end{align}

此时导体内部满足:
\begin{align}
    \left\{\begin{array}{l}
        \nabla ^2\boldsymbol{E}+k^2 \boldsymbol{E} = 0\\
        k = \omega \sqrt{\mu \varepsilon '} = \boldsymbol{\beta} + \mathrm{i} \boldsymbol{\alpha}\\
        \nabla \cdot \boldsymbol{E} = 0\\
        \boldsymbol{H} = \displaystyle \sqrt{\frac{\varepsilon '}{\mu}}\boldsymbol{n}\times \boldsymbol{E}
    \end{array}\right.
\end{align}

这一Helmholtz方程有形式上的平面波解
\begin{align}
    \boldsymbol{E}(\boldsymbol{x}) = \boldsymbol{E}_0 \mathrm{e}^{-\boldsymbol{\alpha} \cdot \boldsymbol{x}} \mathrm{e}^{\mathrm{i}(\boldsymbol{\beta} \cdot \boldsymbol{x}-\omega t)}
\end{align}

这一解中$\mathrm{e}^{-\boldsymbol{\alpha} \cdot \boldsymbol{x}}$的存在表示波幅会随着$x$的增大而衰减.\textbf{垂直入射时}波幅衰减至导体表面原值$\displaystyle \frac{1}{\mathrm{e}}$的传播距离$\delta $称为\textbf{穿透深度},在良导体$\left(\frac{\sigma}{\varepsilon \omega}\gg 1 \right)$情形,穿透距离
\begin{align}
    \delta = \frac{1}{\alpha} = \sqrt{\frac{2}{\omega \mu \sigma}}
\end{align}

由定义得到
\begin{align}
    k^2 = \beta ^2 - \alpha ^2 +2 \mathrm{i}\boldsymbol{\alpha} \cdot \boldsymbol{\beta } = \omega ^2 \mu \left(\varepsilon +\mathrm{i} \frac{\sigma}{\omega}\right)
\end{align}

比较实部和虚部得到
\begin{align}
    &\beta ^2 - \alpha ^2 = \omega ^2 \mu \varepsilon\\
    &\boldsymbol{\alpha} \cdot \boldsymbol{\beta } = \frac{1}{2}\omega \mu \varepsilon
\end{align}

垂直入射时,在良导体中近似有
\begin{align}
    \alpha \approx \beta = \sqrt{\frac{\omega \mu \sigma}{2}}
\end{align}


\section{电磁波的反射与折射}
\subsection{Fresnel公式}

\subsubsection{$\boldsymbol{E} \perp $入射面}
\begin{align}
    &\frac{E'}{E} = -\frac{\mathrm{sin}(\theta - \theta ')}{\mathrm{sin}(\theta + \theta '')}\\
    &\frac{E''}{E} = \frac{2\mathrm{cos} \theta \mathrm{sin} \theta '')}{\mathrm{sin}(\theta + \theta '')}
\end{align}

\subsubsection{$\boldsymbol{E} \parallel $入射面}
\begin{align}
    &\frac{E'}{E} = -\frac{\mathrm{tan}(\theta - \theta '')}{\mathrm{tan}(\theta + \theta '')}\\
    &\frac{E''}{E} = \frac{2\mathrm{cos} \theta \mathrm{sin} \theta '')}{\mathrm{sin}(\theta + \theta '') \mathrm{cos}(\theta - \theta '')}
\end{align}

当$\theta + \theta '' = \frac{\pi}{2}$时,$\frac{E'}{E} = 0$,反射光的平行分量消失,这个特殊的角称为\textbf{Brewster角},满足$\mathrm{tan}\theta = n_21$

\subsection{全反射}
由式\ref{fanshezheshe}知,当$\mathrm{sin}\theta = n_{21}$时,$\theta '' = \pi$,此时折射波沿界面略过.当$\mathrm{sin}\theta > n_{21}$时
\begin{align}
    k''_x = k_x = k\mathrm{sin}\theta ,\ k'' = kn_{21}
\end{align}

故此时$k''_x > k''$,于是
\begin{align}
    k''_z = \sqrt{{k''}^2 - {k''_x}^2} = \mathrm{i}k\sqrt{\mathrm{sin}^2\theta - n_{21}^2} = \mathrm{i}\kappa
\end{align}

此时折射波电场表达式为
\begin{align}
    E'' = E''_0\mathrm{e}^{-\kappa z}\mathrm{e}^{\mathrm{i}(k''_xx-\omegat)}
\end{align}

它的场强沿z轴方向指数衰减,该电磁波只存在于界面附近一薄层内,其厚度
\begin{align}
    \kappa ^{-1} = \frac{1}{k\sqrt{\mathrm{sin}\theta - n_{21}^2}} = \frac{\lambda _1}{2\pi \sqrt{\mathrm{sin} \theta - n_{21}^2}}
\end{align}

\subsection{反射系数}
反射系数定义为反射能流与入射能流之比
\begin{align}
    R = \frac{\bar{\boldsymbol{S}}' \cdot \boldsymbol{e}_n}{\bar{\boldsymbol{S}} \cdot \boldsymbol{e}_n} = \left|\frac{E'_0}{E_0}\right|^2
\end{align}

与之相对的也有折射系数的概念
\begin{align}
    T = \frac{\bar{\boldsymbol{S}}'' \cdot \boldsymbol{e}_n}{\bar{\boldsymbol{S}} \cdot \boldsymbol{e}_n} = \frac{n_2 \mathrm{cos}\theta ''}{n_1 \mathrm{cos}\theta }\left|\frac{E''_0}{E_0}\right|^2
\end{align}

在没有损耗的情况下,总是存在$T + R = 1$.

在良导体表面,有
\begin{align}
    R = \left|\frac{E'_0}{E_0}\right|^2 = 1-2\sqrt{\frac{2\omega \varepsilon _0}{\sigma}}
\end{align}

即电导率越高,反射系数越接近于1.

\section{波导}
\subsection{矩形波导}
取传播方向为$z$轴,在一定频率下,管内电磁波是Helmholtz方程
\begin{align}
    \label{bodaohelmholtz}\left\{\begin{array}{l}
    \nabla ^2 \boldsymbol{E} + k^2 \boldsymbol{E} = 0\\
    k = \displaystyle \omega \sqrt{\mu \varepsilon}\\
    \nabla \cdot \boldsymbol{E} = 0
    \end{array}\right.
\end{align}
电磁波应有传播因子$\mathrm{e}^{\mathrm{i}(k_z z-\omega t)}$,故可以设
\begin{align}
    \boldsymbol{E}(x,y,z) = \boldsymbol{E}(x,y) \mathrm{e}^{\mathrm{i} k_z z}
\end{align}
将其代入式\ref{bodaohelmholtz}中得到
\begin{align}
    \label{juxingbodao}\left(\frac{\partial ^2}{\partial x^2} + \frac{\partial ^2}{\partial y^2} \right)\boldsymbol{E}(x,y) + (k^2 - k^2_z) \boldsymbol{E}(x,y) = 0
\end{align}

设$u(x,y) = X(x)Y(y)$为电磁场的任一直角分量,代入\ref{juxingbodao}得到方程组
\begin{align}
    \displaystyle \frac{\mathrm{d}^2 X}{\mathrm{d}x^2} + k_x^2 X = 0\\
    \displaystyle \frac{\mathrm{d}^2 Y}{\mathrm{d}y^2} + k_y^2 Y = 0
\end{align}

$u(x,y)$的通解为
\begin{align}
    u(x,y) = (C_1\  \mathrm{cos}\ k_xx+D_1\ \mathrm{sin}\ k_xx)(C_2\  \mathrm{cos}\ k_yy+D_2\ \mathrm{sin}\ k_yy)
\end{align}

接下来考虑其边界条件
\begin{align}
    \begin{array}{cc}
    E_y = E_z = 0 & \displaystyle \frac{\partial E_x}{\partial x} = 0(x = 0,a)\\
    E_x = E_z = 0 & \displaystyle \frac{\partial E_y}{\partial y} = 0(y = 0,b)
    \end{array}
\end{align}

解得
\begin{align}
    \left\{\begin{array}{l}
    E_x = A_1\ \mathrm{cos} k_xx\ \mathrm{sin}k_yy\ \mathrm{e}^{\mathrm{i} k_z z}\\
    E_y = A_1\ \mathrm{sin} k_xx\ \mathrm{cos}k_yy\ \mathrm{e}^{\mathrm{i} k_z z}\\
    E_z = A_1\ \mathrm{sin} k_xx\ \mathrm{sin}k_yy\ \mathrm{e}^{\mathrm{i} k_z z}
    \end{array}\right.
\end{align}
其中
\begin{align}
    k_x = \frac{m\pi}{a},\quad k_y = \frac{n\pi}{b}\quad (m,n = 0,1,2,...)
\end{align}

再加上条件$\nabla \cdot \boldsymbol{E} = 0$得到
\begin{align}
    k_xA_1 + k_yA_1 - \mathrm{i}k_zA_3 = 0
\end{align}
即$A_1$,$A_2$和$A_3$中只有两个是独立的.

磁场:
\begin{align}
    \boldsymbol{H} = -\frac{\mathrm{i}}{\omega \mu}\nabla \times \boldsymbol{E}
\end{align}

$E_z = 0$,$H_z \ne 0$的波成为横电波(TE),$H_z = 0$,$E_z \ne 0$的波成为横磁波(TM),按($m,n$)的不同又可以将其分为$\mathrm{TE}_{mn}$波和$\mathrm{TM}_{mn}$波.

\subsection{截止频率}
当激发频率降低至$\displaystyle k<\sqrt{k_x^2 + k_y^2}$,则$k_z$为虚数,传播因子$\mathrm{e}^{\mathrm{i} k_z z}$变为$\mathrm{e}^{-\kappa z}$的衰减因子的形式,此时电磁场不再是沿波导传播的波.能给在波导内传播的波的最低频率$\omega _c$称为该波模的\textbf{截止频率}.前文提到的($m,n$)型波导的截止角频率为
\begin{align}
    \omega_{c,mn} = \frac{\pi}{\sqrt{\mu \varepsilon}}\sqrt{\left(\frac{m}{a}\right)^2 + \left(\frac{n}{b}\right)^2}
\end{align}

若$a>b$则$\mathrm{TE}_10$波的最低截止频率为
\begin{align}
    \frac{1}{2\pi}\omega_{c,10} = \frac{1}{2a\sqrt{\mu \varepsilon}} = \frac{c}{2a}\ \mathrm{(vacuum)}
\end{align}
相应的截止波长为
\begin{align}
    \lambda _{c,10} = 2a
\end{align}

\end{document}