\documentclass[main.tex]{subfiles}
\usepackage[table,xcdraw]{xcolor}
\usepackage{bm}
\usepackage{wallpaper}
\usepackage{booktabs}
\usepackage{array}
\usepackage{ctex}
\usepackage{longtable}
\usepackage{physics}
\usepackage{fancyhdr}
\usepackage{graphicx} % Required for inserting images
\usepackage{float} %指定图片位置
\usepackage{caption}
\usepackage{pdfpages}
\usepackage{color}
\usepackage{rotating}
\usepackage{tabularray}
\usepackage{xeCJK}
\usepackage{verbatim}
\usepackage{type1cm}
\usepackage{tocloft}
\usepackage{multicol}
\usepackage{amssymb}
\usepackage{amsmath}
\usepackage{float}
\usepackage{wrapfig}
\usepackage{diagbox}
\usepackage{multirow}
\usepackage{subfigure}
\usepackage{svg}

\usepackage[hidelinks]{hyperref}
\usepackage[a4paper,left=2.85cm,right=2.85cm,top=2.54cm,bottom=2.54cm]{geometry}
\renewcommand{\normalsize}{\fontsize{14}{16}\selectfont}


\begin{document}

\chapter{矢量分析}

\section{矢量与张量}

\subsection{三维矢量}
$$\boldsymbol{a} = a_i\boldsymbol{e}_i+a_j\boldsymbol{e}_j+a_k\boldsymbol{e}_k$$

矢量运算:
\begin{align}
    &\boldsymbol{a} \cdot \boldsymbol{b} = a_1b_1+a_2b_2+a_3b_3 = \delta _{ij} a^ib^j = a_i b^i\\
    &\boldsymbol{a} \times \boldsymbol{b} = a_2b_3\boldsymbol{e}_1+a_3b_1\boldsymbol{e}_2+a_1b_2\boldsymbol{e}_3 = \varepsilon ^i_{jk}a^jb^k \\
    &\boldsymbol{a} \cdot (\boldsymbol{b} \times \boldsymbol{c}) = \delta _{ij}a^i\varepsilon ^j_{kl}b^kc^l = a_j\varepsilon ^j_{kl}b^kc^l = \varepsilon _{jkl}a^jb^kc^l
\end{align}

\subsection{并矢}
$$\boldsymbol{a}\boldsymbol{b} = a_ib_j\boldsymbol{e}_i\boldsymbol{e}_j$$

有时可以用二阶张量代替并矢:${\substack{\displaystyle\twoheadrightarrow\atop\displaystyle{\mathcal{T}}\\[0.624em]~}} = T_{ij}\boldsymbol{e}_i\boldsymbol{e}_j = a_ib_j\boldsymbol{e}_i\boldsymbol{e}_j$

并矢运算:
\begin{align}
    &(\boldsymbol{a}\boldsymbol{b})\cdot \boldsymbol{c} = a_i b_j c^j \boldsymbol{e}_i \\
    &\boldsymbol{c}\cdot (\boldsymbol{a}\boldsymbol{b}) = c^i a_i b_j \boldsymbol{e}_j \\
    &\boldsymbol{c}\cdot (\boldsymbol{a}\boldsymbol{b}) \cdot \boldsymbol{d} = c^i a_i b_j d^j \\
    &(\boldsymbol{a}\boldsymbol{b}):(\boldsymbol{c}\boldsymbol{d}) = b_i c^i a^j d_j 
\end{align}

\section{微分场论}
\subsection{$\nabla $算符}
在平面直角坐标系中,$\nabla $算符定义为:
\begin{align}
    \nabla = \boldsymbol{e}_x\frac{\partial }{\partial x}+\boldsymbol{e}_y\frac{\partial }{\partial y}+\boldsymbol{e}_z\frac{\partial }{\partial z} 
\end{align}

柱坐标系下定义为:
\begin{align}
    \nabla = \boldsymbol{e}_r\frac{\partial }{\partial r}+\boldsymbol{e}_\theta \frac{1}{r} \frac{\partial }{\partial \theta}+\boldsymbol{e}_z\frac{\partial }{\partial z} 
\end{align}

球坐标系下定义为:
\begin{align}
    \nabla = \boldsymbol{e}_r\frac{\partial }{\partial r}+\boldsymbol{e}_\theta \frac{1}{r} \frac{\partial }{\partial \theta}+\boldsymbol{e}_\varphi \frac{1}{r\mathrm{sin}\theta }\frac{\partial }{\partial \varphi} 
\end{align}

\subsection{梯度}
定义线元$\mathrm{d}\boldsymbol{l} = \mathrm{d}x\boldsymbol{e}_x+\mathrm{d}y\boldsymbol{e}_y+\mathrm{d}z\boldsymbol{e}_z$

对于一个\textbf{标量场}$\varphi $,可做全微分
\begin{align}
    \mathrm{d}\varphi = \frac{\partial \varphi }{\partial x}\mathrm{d}x+\frac{\partial \varphi }{\partial y}\mathrm{d}y+\frac{\partial \varphi }{\partial z}\mathrm{d}z
\end{align}

于是可以定义该标量场$\varphi $的梯度为
\begin{align}
    \mathrm{grad}\ \varphi  =\boldsymbol{e}_x\frac{\partial \varphi }{\partial x}+\boldsymbol{e}_y\frac{\partial \varphi }{\partial y}+\boldsymbol{e}_z\frac{\partial \varphi }{\partial z}
\end{align}

利用$\nabla $算符,梯度可以表示为
\begin{align}
    &\mathrm{grad}\ \varphi  = \nabla \varphi \\
    &\mathrm{d}\varphi = \nabla \varphi \cdot \mathrm{d}\boldsymbol{l}
\end{align}

\subsection{散度}
对于三维矢量场$\boldsymbol{f}(x,y,z)$,设曲面S围着体积$\Delta V$,当$\Delta V \to 0$时,$\boldsymbol{f}$对S的通量与$\Delta V$之比的极限称为$\boldsymbol{f}$的散度:
\begin{align}
    \mathrm{div}\ \boldsymbol{f} = \lim_{\Delta V \to 0} \frac{\displaystyle \oint \boldsymbol{f}\cdot \mathrm{d}\boldsymbol{S}}{\Delta V}
\end{align}

特别地,在直角坐标系下,可以推导出:
\begin{align}
    \mathrm{div}\ \boldsymbol{f} = \nabla \cdot \boldsymbol{f}
\end{align}

要注意的是,在柱坐标系或球坐标系下$\mathrm{div}\ \boldsymbol{f}$也可以写作$\nabla \cdot \boldsymbol{f}$,但只是作为一种写法使用,数学上二者并不相等.在柱坐标系下:
\begin{align}
    \mathrm{div}\ \boldsymbol{f} = \frac{1}{r}\frac{\partial (rf_r)}{\partial r}+\frac{1}{r} \frac{\partial f_\theta}{\partial \theta}+\frac{\partial f_z}{\partial z}
\end{align}

在球坐标系下:
\begin{align}
    \mathrm{div}\ \boldsymbol{f} = \frac{1}{r^2}\frac{\partial (r^2f_r)}{\partial r}+\frac{1}{r\mathrm{sin}\theta } \frac{\partial (\theta f_\theta)}{\partial \theta}+\frac{1}{r\mathrm{sin}\theta }\frac{\partial f_z}{\partial z}
\end{align}

\subsection{旋度}
对于三维矢量场$\boldsymbol{f}(x,y,z)$,设闭合曲线L围着体积$\Delta S$,当$\Delta S \to 0$时,$\boldsymbol{f}$对L的通量与$\Delta S$之比的极限称为$\boldsymbol{f}$的散度:
\begin{align}
    (\mathrm{rot}\ \boldsymbol{f})_n = \lim_{\Delta S \to 0} \frac{\displaystyle \oint \boldsymbol{f}\cdot \mathrm{d}\boldsymbol{l}}{\Delta S}
\end{align}

同样地,在直角坐标系下有:
\begin{align}
    \mathrm{rot}\ \boldsymbol{f} = \nabla \times \boldsymbol{f}
\end{align}

\subsection{关于旋度与散度的重要定理}
\subsubsection{微分}
\begin{align}
    &\nabla \times (\nabla \varphi ) = 0\\
    &\nabla \cdot (\nabla \times \boldsymbol{f} ) = 0\\
    &\nabla \varphi \to \mathrm{vector}\\
    &\nabla \cdot (\nabla \varphi ) = \nabla ^2 \varphi = \frac{\partial ^2 \varphi}{\partial x^2}+\frac{\partial ^2 \varphi}{\partial y^2}+\frac{\partial ^2 \varphi}{\partial z^2}
\end{align}

\subsubsection{积分变换式}
由散度、旋度定义可以得到
\begin{align}
    \label{jifenbianhuan1}&\oint_{S}^{}\boldsymbol{f}\cdot \mathrm{d}\boldsymbol{S} = \int_{V}^{} \mathrm{div}\ \boldsymbol{f}\mathrm{d}V\\
    \label{jifenbianhuan2}&\oint_{L}^{}\boldsymbol{f}\cdot \mathrm{d}\boldsymbol{l} = \int_{S}^{} \mathrm{rot}\ \boldsymbol{f}\mathrm{d}\boldsymbol{S}
\end{align}

\subsubsection{$\nabla $算符运算公式}
下方公式中$\varphi ,\psi $代表标量场、$\boldsymbol{f},\boldsymbol{g}$代表矢量场.
\begin{align}
    &\nabla (\varphi \psi ) = \varphi \nabla \psi + \psi \nabla \varphi\\
    &\nabla \cdot (\varphi \boldsymbol{f}) = (\nabla \varphi )\cdot \boldsymbol{f} + \varphi \nabla \cdot \boldsymbol{f}\\
    &\nabla \times (\varphi \boldsymbol{f}) = (\nabla \varphi )\times \boldsymbol{f} + \varphi \nabla \times \boldsymbol{f}\\
    \label{nabla4}&\nabla \cdot (\boldsymbol{f}\times \boldsymbol{g}) = (\nabla \times \boldsymbol{f})\cdot \boldsymbol{g} - \boldsymbol{f} \cdot (\nabla \times \boldsymbol{g})\\
    &\nabla \times (\boldsymbol{f} \times \boldsymbol{g}) = (\boldsymbol{g} \cdot \nabla) \boldsymbol{f}+(\nabla \cdot \boldsymbol{g})\boldsymbol{f}-(\boldsymbol{f} \cdot \nabla)\boldsymbol{g} - (\nabla \cdot \boldsymbol{f})\boldsymbol{g}\\
    &\nabla(\boldsymbol{f}\cdot \boldsymbol{g}) = \boldsymbol{f}\times (\nabla \times \boldsymbol{g}) + (\boldsymbol{f} \cdot \nabla )\boldsymbol{g} + \boldsymbol{g}\times (\nabla \times \boldsymbol{f})+(\boldsymbol{g} \cdot \nabla)\boldsymbol{f}\\
    \label{nabla7}&\nabla \times (\nabla \times \boldsymbol{f}) = \nabla (\nabla \cdot \boldsymbol{f}) - \nabla ^2 \boldsymbol{f}
\end{align}
\end{document}