\documentclass[main.tex]{subfiles}
\usepackage[table,xcdraw]{xcolor}
\usepackage{bm}
\usepackage{wallpaper}
\usepackage{booktabs}
\usepackage{array}
\usepackage{ctex}
\usepackage{longtable}
\usepackage{physics}
\usepackage{fancyhdr}
\usepackage{graphicx} % Required for inserting images
\usepackage{float} %指定图片位置
\usepackage{caption}
\usepackage{pdfpages}
\usepackage{color}
\usepackage{rotating}
\usepackage{tabularray}
\usepackage{xeCJK}
\usepackage{verbatim}
\usepackage{type1cm}
\usepackage{tocloft}
\usepackage{multicol}
\usepackage{amssymb}
\usepackage{amsmath}
\usepackage{float}
\usepackage{wrapfig}
\usepackage{diagbox}
\usepackage{multirow}
\usepackage{subfigure}
\usepackage{svg}
\usepackage[hidelinks]{hyperref}
\usepackage[a4paper,left=2.85cm,right=2.85cm,top=2.54cm,bottom=2.54cm]{geometry}
\renewcommand{\normalsize}{\fontsize{14}{16}\selectfont}


\begin{document}

\chapter{狭义相对论}
\section{Lorentz变换}
\subsection{相对论的基本原理}
Einstein于1905年提出:

(1)光速不变原理.可以记$c=1(3\times 10^8 \mathrm{m}/ \mathrm{s})$

(2)相对性原理:惯性系是等价的.体现了四维矢量的协变性.

\begin{wrapfigure}{r}{7cm}
	\centering
    \includesvg[width=0.9\linewidth]{xiangduilun1.svg}
\end{wrapfigure}

\subsection{四维矢量与协变性}

在牛顿力学中,一个三维矢量$\overrightarrow{PQ}$,其模长$|\overrightarrow{PQ}|^2 = x^2 + y^2 + z^2$,且在任意惯性系下均成立,这就是其牛顿力学的协变性.

四维矢量则是在三个空间维度的基础上引入时间轴,变为$r=(x,y,z,ct)$.此时$|\overrightarrow{PQ}|^2 = x^2 + y^2 + z^2 - c^2t^2$.令$t$为第0分量,我们可以定义一个张量场$g_{ij}$,满足$|\overrightarrow{PQ}|^2 = g_{ij} r^i r^j$,即
\begin{align}
    g_{ij} = \left\{\begin{array}{cl}
  1& \mathrm{when}\ i  = j \ne 0\\
  -1& \mathrm{when}\ i  = j  = 0\\
  0&\mathrm{when}\ i \ne j
\end{array}\right.
\end{align}

$g_{ij}$被称为(狭义相对论中的)\textbf{度规}.此时
\begin{align}
    v_i = g_{ij} v^i = \left\{\begin{array}{cl}
    v^1 &\mathrm{when}\ i = 1\\
    v^2 &\mathrm{when}\ i = 2\\
    v^3 &\mathrm{when}\ i = 3\\
    -v^0 &\mathrm{when}\ i = 0\\
    \end{array}\right.
\end{align}

故$v_iv^i = {v^1}^2+{v^2}^2+{v^3}^2-{v^0}^2$

或者可以令$r = (x,y,z,\mathrm{i}ct)$此时便无需引入度规.

\subsection{间隔不变性}
由相对性原理得到,当
\begin{align}
    x^2 + y^2 + z^2 -c^2t^2 = 0
\end{align}

时,亦有
\begin{align}
    x'^2 + y'^2 + z'^2 -c^2t'^2 = 0
\end{align}

故一定存在
\begin{align}
    x^2 + y^2 + z^2 -c^2t^2 = A(x'^2 + y'^2 + z'^2 -c^2t'^2 )
\end{align}\

由于空间中不存在特定方向,故亦存在
\begin{align}
    x'^2 + y'^2 + z'^2 -c^2t'^2 = A(x^2 + y^2 + z^2 -c^2t^2)
\end{align}

解得
\begin{align}
    \label{jiangebubianxing}\left\{\begin{array}{l}
    A = 1\\
    x^2 + y^2 + z^2 -c^2t^2 = x'^2 + y'^2 + z'^2 -c^2t'^2
    \end{array}\right.
\end{align}

令
\begin{align}
    &s^2 = c^2t^2 - (x^2 + y^2 +z^2)\\
    &s'^2 = c^2t'^2 - (x'^2 + y'^2 +z'^2)
\end{align}

则可以把关系式\ref{jiangebubianxing}简写为
\begin{align}
    s^2 = s'^2
\end{align}

称为\textbf{间隔不变性},表示两事件的间隔不随参考系变换而改变.

\subsection{Lorentz变换的引入}

\begin{wrapfigure}{r}{7cm}
	\centering
    \includesvg[width=0.9\linewidth]{xiangduilun2.svg}
\end{wrapfigure}

为简单考虑,选取$x$轴和$x'$轴都沿$\Sigma '$相对于$\Sigma $的运动方向,此时可以假设
\begin{align}
    \label{Lorentzbianhuanshi}\begin{array}{c}
    x' = a_{11}x+a_{12}ct\\
    y' = y\\
    z' = z\\
    ct' = a_{21}x + a_{22}ct
    \end{array}
\end{align}

或是写为
\begin{align}
    \begin{bmatrix}
      a_{11}&a_{12} \\
      a_{21}&a_{22}
    \end{bmatrix}  
    \begin{bmatrix}
     x\\
    ct
    \end{bmatrix} = 
    \begin{bmatrix}
     x'\\
    ct'
    \end{bmatrix}
\end{align}

代入时空间隔表达式\ref{jiangebubianxing}得到
\begin{align}
    \begin{array}{c}
        a_{11}^2 - a_{21}^2 = 1\\
        a_{11}a_{12} - a_{21}a_{22} = 0\\
        a_{12}^2 - a_{22}^2 = -1
    \end{array}
\end{align}

解得
\begin{align}
    a_{11} = \sqrt{1+a_{21}^2},\quad a_{22} = \sqrt{1+ a_{12}^2},\quad a_{12} = a_{21}
\end{align}

对于$\Sigma '$系来说,$O'$点永远$x' = 0$,而对于$\Sigma $系,$O$点坐标为$x = vt$,故由\ref{Lorentzbianhuanshi}第一式得到
\begin{align}
    a_{11}vt + a_{12}ct = 0
\end{align}

即
\begin{align}
    \frac{a_{12}}{a_{11}} = -\frac{v}{c}
\end{align}

进而推导出所有四个系数的值:
\begin{align}
    &a_{11} = a_{22} = \frac{1}{\displaystyle \sqrt{1-\frac{v^2}{c^2}}}\\
    &a_{12} = a_{21} = \frac{\displaystyle -\frac{v}{c}}{\displaystyle \sqrt{1-\frac{v^2}{c^2}}}
\end{align}

最终得到Lorentz变换公式
\begin{align}
    \label{Lorentzbianhuan}\begin{array}{c}
    x' = \frac{\displaystyle x' - vt}{\displaystyle \sqrt{1-\frac{v^2}{c^2}}}\\
    y' = y\\
    z' = z\\
    t' = \frac{\displaystyle t - \frac{v}{c^2}x}{\displaystyle \sqrt{1- \frac{v^2}{c^2}}}
    \end{array}
\end{align}

或者用矩阵形式写成
\begin{align}
    \label{Lorentzjuzhen}\begin{bmatrix}
  \frac{1}{\sqrt{1-\frac{v^2}{c^2} } } & 0 & 0 & -\frac{v}{\sqrt{1-\frac{v^2}{c^2}}} \\
  0& 1 & 0 &0 \\
 0 & 0 & 1 & 0\\
  -\frac{v/c^2}{\sqrt{1-\frac{v^2}{c^2}}} & 0 & 0 &\frac{1}{\sqrt{1-\frac{v^2}{c^2}}} 
\end{bmatrix}
\begin{bmatrix}
 x\\
 y\\
 z\\
t
\end{bmatrix}
=
\begin{bmatrix}
 x'\\
 y'\\
 z'\\
t'
\end{bmatrix}
\end{align}

由于$\Sigma '$系与$\Sigma $系是等价的,因此若$\Sigma '$系相对$\Sigma $系以$\boldsymbol{v}$的速度运动,则$\Sigma $系相对$\Sigma '$系以$-\boldsymbol{v}$的速度运动.因此将Lorentz变换中的$v$全部改为$-v$即可得到逆变换式
\begin{align}
    \begin{array}{c}
    x' = \frac{\displaystyle x' + vt}{\displaystyle \sqrt{1-\frac{v^2}{c^2}}}\\
    y' = y\\
    z' = z\\
    t' = \frac{\displaystyle t + \frac{v}{c^2}x}{\displaystyle \sqrt{1- \frac{v^2}{c^2}}}
    \end{array}
\end{align}

\section{相对论的时空理论}
\subsection{尺缩效应}
\begin{wrapfigure}{r}{8cm}
	\centering
    \includesvg[width=0.9\linewidth]{xiangduilun3.svg}
\end{wrapfigure}

应用Lorentz变换式得到
\begin{align}
    &x' = \frac{\displaystyle x - vt}{\displaystyle \sqrt{1- \frac{v^2}{c^2}}}\\
    &\Delta x' = \frac{\displaystyle \Delta x - v\Delta t}{\displaystyle \sqrt{1- \frac{v^2}{c^2}}}
\end{align}

由于要求同时观测,因此$\Delta t =0$,即
\begin{align}
    \Delta x = \Delta x ' \sqrt{1- \frac{v^2}{c^2}}
\end{align}

该式表明在$\Sigma $系下观测到物体长度$\Delta x$要短于$\Sigma '$系下观测到物体静止长度$\Delta x'$

\subsection{钟慢效应}
\begin{figure}[h]
    \centering
    \includesvg[width=1\linewidth]{xiangduilun4.svg} 
\end{figure}

对于静止坐标系$\Sigma '$而言,两事件发生于同一地点$\boldsymbol{x}'$,设时间间隔为$\Delta \tau$,则时空间隔为
\begin{align}
    \Delta s'^2 = c^2 \Delta \tau ^2
\end{align}

而对于另一坐标系$\Sigma$而言,该物体以$\Delta v$运动,两事件发生于不同地点,设两事件发生的地点间隔$\Delta \boldsymbol{x}$,时间间隔$\Delta t$,则时空间隔为
\begin{align}
    \Delta s^2 = c^2 \Delta t^2 -(\Delta \boldsymbol{x})^2
\end{align}

由间隔不变性有
\begin{align}
    c^2 \Delta t^2 -(\Delta \boldsymbol{x})^2 = c^2 \Delta \tau ^2
\end{align}

$|\Delta \boldsymbol{x}|/\Delta t = v$为物体相对$\Sigma$的运动速度,因此有
\begin{align}
    \Delta t = \frac{\displaystyle \Delta \tau }{\displaystyle \sqrt{1- \frac{v^2}{c^2}}}
\end{align}

即所观测物体运动速度越大,观察到的其内部物理过程进行得越缓慢.

\subsection{速度变换公式}
由Lorentz变换公式\ref{Lorentzbianhuan}可得
\begin{align}
    u'_x = \frac{\mathrm{d}x'}{\mathrm{d}t'} = \frac{\displaystyle (\mathrm{d}x - v\mathrm{d}t)/\sqrt{1 - \frac{v^2}{c^2}}}{\displaystyle (\mathrm{d}t +\frac{v}{c^2}\mathrm{d}x)/\sqrt{1-\frac{v^2}{c^2}}} = \frac{\displaystyle u_x -v}{\displaystyle 1-\frac{vu_x}{c^2}}
\end{align}

同理可得
\begin{align}
    &u'_y = \frac{\displaystyle u_y\sqrt{1-\frac{v^2}{c^2}}}{\displaystyle 1-\frac{vu_x}{c^2}}\\
    &u'_z = \frac{\displaystyle u_z\sqrt{1-\frac{v^2}{c^2}}}{\displaystyle 1-\frac{vu_x}{c^2}}
\end{align}

\section{相对论理论的四维形式}
\subsection{物理量按空间变换性质的重新分类}
标量:坐标系变换下与取向无关,故不变的物理量,例如质量、电荷.在两坐标系下标量有
\begin{align}
    u' = u
\end{align}

矢量:在两个参考系下满足
\begin{align}
    v'_i = a_{ij}v_j
\end{align}
的量,例如速度、位矢、相对论中的四维矢量等等.$\nabla $算符在该定义下也可以认为是矢量.

二阶张量(并矢):当空间变换时满足
\begin{align}
    T'_{ij} = a_{ik}a_{im}T_{km}
\end{align}
的量(两个分量分别进行变换).如电磁场应力张量,电四极矩等.

二阶张量又可以进一步分类为

\textbf{对称张量}:$T_{ij} = T_{ji}$

\textbf{反对称张量}:$T_{ij} = -T_{ji}$

不论是对称还是反对称张量,变换后仍具有对称与反对称性.

\textbf{迹}:$\mathrm{Tr}(T_{ij}) = T_{ii}$,对称张量之迹是一个标量.电四极矩是一个无迹对称张量,只有5个独立分量.

\subsection{Lorent变换的四维形式}
采用虚数单位的形式,沿$x$轴方向的特殊Lorentz变换式 \ref{Lorentzbianhuan} 的变换矩阵为
\begin{align}
    a = \begin{bmatrix}
    \gamma & 0 & 0 & \mathrm{i} \beta \gamma \\
    0 & 1 & 0 & 0\\
    0 & 0 & 1 & 0\\
    -\mathrm{i} \beta \gamma & 0 & 0 & \gamma 
    \end{bmatrix}
\end{align}

即式\ref{Lorentzjuzhen}.式中$\beta = \displaystyle \frac{v}{c}$,$\gamma = \displaystyle \frac{1}{\sqrt{1- \frac{v^2}{c^2}}}$

逆变换矩阵为
\begin{align}
    a^{-1} = \begin{bmatrix}
    \gamma & 0 & 0 & -\mathrm{i} \beta \gamma \\
    0 & 1 & 0 & 0\\
    0 & 0 & 1 & 0\\
    \mathrm{i} \beta \gamma & 0 & 0 & \gamma 
    \end{bmatrix} = a^{T}
\end{align}

满足正交条件$a^{T}a = I$

\subsubsection{四维协变量}
\textbf{位矢的四维矢量}为
\begin{align}
    r_{\mu} = (x,y,z,\mathrm{i}ct)
\end{align}

\textbf{速度的四维矢量}为
\begin{align}
    \label{4-sudu}U_{\mu} = \gamma _{u}(u_{1},u_{2},u_{3},\mathrm{i}c)
\end{align}
系数$\gamma _{u}$的出现是由于当坐标系变换时$\mathrm{d}x_{i}$按四维矢量的分量变换,但$\mathrm{d}t$也发生改变导致的.
\begin{align}
    \gamma _{u} = \frac{\mathrm{d}t}{\mathrm{d}\tau} = \frac{1}{\displaystyle \sqrt{1- \frac{u^2}{c^2}}}
\end{align}

\textbf{相位的四维矢量}为
\begin{align}
    k_{\mu} = \left(k_1, k_2, k_3,\mathrm{i} \displaystyle \frac{\omega}{c}\right)
\end{align}
这是由于相位只是计数问题,不随参考系而变,故
\begin{align}
    \phi = \phi ' = \mathrm{invariant}
\end{align}
即
\begin{align}
    \boldsymbol{k} \cdot \boldsymbol{x} -\omega t = \boldsymbol{k}' \cdot \boldsymbol{x}' -\omega ' t' = \mathrm{invariant}
\end{align}
由于$\boldsymbol{x}$与$\mathrm{i}ct$合成为四维矢量$x_{\mu}$,故可以设$\boldsymbol{k}$与$\mathrm{i}\displaystyle \frac{\omega}{c}$合成为四维矢量$k_{\mu}$这样便有
\begin{align}
    k_{\mu} x_{\mu} = k_{\mu}' x_{\mu}' = \mathrm{invariant}
\end{align}

\section{电动力学中的协变量}
\subsection{电流密度矢量}
电荷量与其运动速度无关,故
\begin{align}
    Q = \int \rho \mathrm{d}V = \int \rho \sqrt{\displaystyle 1 - \frac{u^2}{c^2}}\mathrm{d}V' = \int \rho '\mathrm{d}V'
\end{align}
得到
\begin{align}
    \rho = \displaystyle \frac{\rho '}{\displaystyle \sqrt{1-\frac{u^2}{c^2}}} = \gamma _{u}\rho '
\end{align}

由于
\begin{align}
    \boldsymbol{J} = \rho \boldsymbol{u} = \gamma _u \rho '\boldsymbol{u}
\end{align}
因此可以引入
\begin{align}
    J_0 = \mathrm{i}c \rho
\end{align}
使得
\begin{align}
    J_{\mu} = \rho ' U_{\mu}
\end{align}

电荷守恒定律用四维形式就可以简单表示为
\begin{align}
    \frac{\partial J_{\mu}}{\partial x_{\mu}} = 0
\end{align}

\subsection{四维势矢量}
用势表示的电动力学基本方程组\ref{d'Alembert}在Lorentz规范下为
\begin{align}
    \label{d'Alembert2}\begin{array}{c}
    \nabla ^2 \boldsymbol{A} - \displaystyle \frac{1}{c^2}\frac{\partial ^2 \boldsymbol{A}}{\partial t^2} = -\mu _0 \boldsymbol{J}\\
    \nabla ^2 \varphi  - \displaystyle \frac{1}{c^2}\frac{\partial ^2 \varphi }{\partial t^2} = -\frac{\rho}{\varepsilon _0}\\
    \left(\nabla \cdot \boldsymbol{A} +\displaystyle \frac{1}{c^2}\frac{\partial \varphi }{\partial  t}  = 0\right)
    \end{array}    
\end{align}

定义算符
\begin{align}
    \Box \equiv \nabla ^2 -\frac{1}{c^2}\frac{\partial ^2}{\partial t^2}  = \frac{\partial }{\partial  x_{\mu}} \frac{\partial }{\partial  x_{\mu}} 
\end{align}
则\ref{d'Alembert2}可以改写为
\begin{align}
    \begin{array}{c}
        \label{siweid'Alembert}\Box \boldsymbol{A} = -\mu _0\boldsymbol{J}\\
         \Box \varphi = -\mu _0 c^2 \rho
    \end{array}
\end{align}

可以将$\boldsymbol{A}$和$\varphi$合并为一个四维矢量
\begin{align}
    A_{\mu} = \left(\boldsymbol{A},\frac{\mathrm{i}}{c}\varphi \right)
\end{align}

这样\ref{siweid'Alembert}就可以合写为
\begin{align}
    \Box A_{\mu} = -\mu_0J_{\mu}
\end{align}

\subsection{电磁场张量}
电磁场$\boldsymbol{E}$和$\boldsymbol{B}$用势表出为
\begin{align}
    \begin{array}{c}
         \boldsymbol{B} = \nabla \times \boldsymbol{A}\\
         \boldsymbol{E} = -\nabla \varphi - \frac{\partial \boldsymbol{A}}{\partial t}
    \end{array}
\end{align}
其分量为
\begin{align}
    \begin{array}{c}
        B_1 = \displaystyle \frac{\partial A_3}{\partial x_2} - \displaystyle \frac{\partial A_2}{\partial x_3},\ B_2 = \displaystyle \frac{\partial A_1}{\partial x_3} - \displaystyle \frac{\partial A_3}{\partial x_1},\ B_3 = \displaystyle \frac{\partial A_2}{\partial x_1} - \displaystyle \frac{\partial A_1}{\partial x_2}\\
        E_i = \mathrm{i}c\left(\displaystyle \frac{\partial A_0}{\partial x_i} - \displaystyle \frac{\partial A_i}{\partial x_0} \right)
    \end{array} 
\end{align}

引入一个反对称四维张量
\begin{align}
    F_{\mu \nu} = \frac{\partial A_{\nu}}{\partial \x_{\mu}} - \frac{\partial A_{\mu}}{\partial \x_{\nu}} = 
    \begin{bmatrix}
    0 & \frac{\mathrm{i}}{c}E_1 & \frac{\mathrm{i}}{c}E_2 & \frac{\mathrm{i}}{c}E_3\\
    -\frac{\mathrm{i}}{c}E_1  & 0 & B_3 & -B_2 \\
    -\frac{\mathrm{i}}{c}E_2 & -B_3 & 0 & B_1\\
    -\frac{\mathrm{i}}{c}E_3 & B_2 & -B_1 & 0
    \end{bmatrix}
\end{align}

或者按课本上的写法,将$A_{\mu}$中的$\displaystyle \frac{\mathrm{i}}{c} \varphi$作为第四项,则$F_{\mu \nu}$变为\
\begin{align}
    F_{\mu \nu} = \begin{bmatrix}
    0 & B_3 & -B_2 &-\frac{\mathrm{i}}{c}E_1\\
    -B_3 & 0 & B_1&-\frac{\mathrm{i}}{c}E_2 \\
    B_2 & -B_1 & 0&-\frac{\mathrm{i}}{c}E_3 \\
    \frac{\mathrm{i}}{c}E_1 & \frac{\mathrm{i}}{c}E_2 & \frac{\mathrm{i}}{c}E_3 & 0
    \end{bmatrix}
\end{align}

于是,Maxwell方程组中的一对方程
\begin{align}
    \left\{\begin{array}{l}
    \nabla \cdot \boldsymbol{E} = \displaystyle \frac{\rho}{\varepsilon _0}\\
    \nabla \times \boldsymbol{B} = \displaystyle \mu_0 \varepsilon _0 \frac{\partial \boldsymbol{E}}{\partial t} + \mu_0 \boldsymbol{J}
    \end{array}\right.
\end{align}
就可以简写为
\begin{align}
    \frac{\partial F_{\mu \nu}}{\partial x_{\nu}} = \mu _0J_{\mu}
\end{align}
另一对方程
\begin{align}
    \left\{\begin{array}{l}
    \nabla \cdot \boldsymbol{B} = 0\\
    \nabla \times \boldsymbol{E} = \displaystyle - \frac{\partial \boldsymbol{B}}{\partial t}
    \end{array}\right.
\end{align}
就可以简写为
\begin{align}
    \frac{\partial F_{\mu \nu}}{\partial x_{\lambda}} + \frac{\partial F_{\nu \lambda}}{\partial x_{\mu}} + \frac{\partial F_{\lambda \mu}}{\partial x_{\nu}} = 0
\end{align}

由张量变换关系$F_{\mu \nu}' = a_{\mu \lambda}a_{\nu \tau}F_{\lambda \tau}$可以得到电场与磁场的变换关系式:
\begin{align}
    \begin{array}{cc}
        \boldsymbol{E}'_\parallel = \boldsymbol{E}_\parallel  & \boldsymbol{B}'_\parallel = \boldsymbol{B}_\parallel \\
        \boldsymbol{E}'_\perp  = \gamma (\boldsymbol{E}+ \boldsymbol{v} \times \boldsymbol{B})_\perp  & \boldsymbol{B}'_\perp  = \gamma (\boldsymbol{B}-\displaystyle \frac{\boldsymbol{v}}{c^2} \times \boldsymbol{B})_\perp
    \end{array}
\end{align}

\section{相对论力学}
\subsection{能量-动量四维矢量}
牛顿力学中有
\begin{align}
    \boldsymbol{F} = \frac{\mathrm{d}\boldsymbol{p}}{\mathrm{d}t},\quad \boldsymbol{p} = m\boldsymbol{v}
\end{align}

但$\boldsymbol{v}$在相对论中不是一个协变量,因此可以引入式\ref{4-sudu}以定义\textbf{四维动量矢量}
\begin{align}
    p_{\mu} = m_0U_{\mu}
\end{align}
其空间与时间分量
\begin{align}
    \begin{array}{c}
    \boldsymbol{p} = \gamma m_0 \boldsymbol{v} = m\boldsymbol{v}\\
    p_0 = \mathrm{i}c\gamma m_0 = \displaystyle \frac{\mathrm{i}}{c}\gamma m_0 c^2 = \frac{\mathrm{i}}{c}\left(m_0 c^2 + \frac{1}{2}m_0v^2+...\right) = \frac{\mathrm{i}}{c}W
    \end{array}
\end{align}

由四维矢量$p_{\mu}$可构成不变量
\begin{align}
    p_{\mu}p^{\mu} = \boldsymbol{p}^2 -  \frac{W^2}{c^2} = \mathrm{invariant}
\end{align}
在静止系内,$\boldsymbol{p} = 0$,$W = m_0c^2$,因此
\begin{align}
    W = \sqrt{p^2 c^2 + m_0^2 c^4}
\end{align}

\subsection{四维力矢量}
定义外界对物体的作用可以用一个四维力矢量$K_{\mu}$描述,则
\begin{align}
    K_{\mu} = \frac{\mathrm{d}p_{\mu}}{\mathrm{d}\tau}
\end{align}
其时间分量
\begin{align}
    K_0 = \frac{\mathrm{i}}{c} \frac{\mathrm{d}W}{\mathrm{d}\tau} = \frac{\mathrm{i}}{c} \frac{\mathrm{d}}{\mathrm{d}\tau} = \frac{\mathrm{i}}{c} \frac{c^2}{W}\boldsymbol{p} \cdot \frac{\mathrm{d}\boldsymbol{p}}{\mathrm{d}\tau} = \frac{\mathrm{i}}{c} \boldsymbol{K} \cdot \boldsymbol{v}
\end{align}

因此,作用在速度$v$物体上的四维力矢量为
\begin{align}
    K_{\mu} = \left(\boldsymbol{K},\frac{\mathrm{i}}{c}\boldsymbol{K} \cdot \boldsymbol{v} \right)
\end{align}

相对论协变的力学方程为
\begin{align}
    \left\{\begin{array}{l}
         \boldsymbol{K} = \displaystyle \frac{\mathrm{d} \boldsymbol{p}} {\mathrm{d} \tau} = \displaystyle \gamma \frac{\mathrm{d} \boldsymbol{p}} {\mathrm{d} t} \\
          \boldsymbol{K} \cdot \boldsymbol{v} = \displaystyle \frac{\mathrm{d} W} {\mathrm{d} \tau} = \displaystyle \gamma \frac{\mathrm{d} W} {\mathrm{d} t}
    \end{array}\right.
\end{align}

定义力为
\begin{align}
    \boldsymbol{F} = \sqrt{1 - \frac{v^2}{c^2}} \boldsymbol{K} = \frac{\boldsymbol{K}}{\gamma}
\end{align}
则相对论力学方程可以写为
\begin{align}
    \left\{\begin{array}{l}
         \boldsymbol{F} = \displaystyle \frac{\mathrm{d} \boldsymbol{p}} {\mathrm{d} t} \\
          \boldsymbol{F} \cdot \boldsymbol{v} = \displaystyle \frac{\mathrm{d} W} {\mathrm{d} t}
    \end{array}\right.
\end{align}

\end{document}